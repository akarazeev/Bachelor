\chapter*{Введение}                         % Заголовок
\addcontentsline{toc}{chapter}{Введение}    % Добавляем его в оглавление

\textbf{Объектом исследования} являются методы для экстраполяции временных рядов и поиска выбросов в данных, а также способы их реализации и области применения.

\textbf{Предметом исследования} является анализ существующих алгоритмов для обработки данных.

\textbf{Методы исследования} — анализ предметной области, анализ реализованных методов, написание программного кода и извлечение полезной информации из данных

\textbf{Актуальность выбранной темы} — объём данных, которые появляются каждый день, растёт с экспоненциальной скоростью. Необходимо уметь работать с большими объёмами данных, чтобы получать из этого пользу. А для этого необходимо использовать актуальные алгоритмы и подходы, которые уже имеются. Реализация современных алгоритмов поиска аномалий в данных может помочь в разных областях, таких как выявление отклонений в здоровье пациента, мошенничества в банковской сфере и др. Спрос на подобные сервисы в настоящее время высок и растёт с каждым днём.

\textbf{Объем и структура работы.} Работа состоит из~введения и пяти глав.
Полный объём работы составляет
\formbytotal{TotPages}{страниц}{у}{ы}{}, включая
\formbytotal{totalcount@figure}{рисун}{ок}{ка}{ков} и
\formbytotal{totalcount@table}{таблиц}{у}{ы}{}.   Список литературы содержит
\formbytotal{citenum}{наименован}{ие}{ия}{ий}.
