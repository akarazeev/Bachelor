\chapter{Анализ предметной области} \label{ch:ch1}

\section{Постановка задачи} \label{sec:ch1/sec1}

В настоящее время существует множество подходов к определению аномалий в данных. Все они хорошо применимы в своей области. Именно поэтому следует в первую очередь проанализировать существующие алгоритмы и их сферы применимости. После чего предлагается объединить их в один сервис, чтобы упростить жизнь потенциальному пользователю, которому необходимо будет проанализировать большой объём данных.

\todo{
Anomaly detection is the process of finding instances in a data set which are different from the majority of the data. It is used in a variety of application domains. In the network security domain it is referred to as intrusion detection, the process of finding outlying instances in network traffic or in system calls of computers indicating compromised systems. In the forensics domain, anomaly detection is also heavily used and known as outlier detection, fraud detection, misuse detection or behavioral analysis. Applications include the detection of payment fraud analyzing credit card transactions, the detection of business crime analyzing financial transactional data or the detection of data leaks from com- pany servers in data leakage prevention (DLP) systems. Furthermore, anomaly detection has been applied in the medical domain as well by monitoring vital functions of patients and it is used for detecting failures in complex systems, for example during space shuttle launches.
However, all of these application domains have in common, that normal be- havior needs to be identified and outlying instances should be detected. This leads to two basic assumptions for anomaly detection:
– anomalies only occur very rarely in the data and
– their features do differ from the normal instances significantly.}

\clearpage

\section{Объект, предмет и методы исследования} \label{sec:ch1/sec2}

\textbf{Объектом исследования} являются методы для поиска аномалий в данных, а также способы их реализации и области применения.

\textbf{Предметом исследования} является анализ существующих алгоритмов для обработки данных.

\textbf{Методы исследования} — анализ предметной области, анализ реализованных методов, написание программного кода и извлечение полезной информации из данных

\clearpage

\section{Актуальность выбранной темы} \label{sec:ch1/sec3}

Объём данных, которые появляются каждый день, растёт с экспоненциальной скоростью. Необходимо уметь работать с большими объёмами данных, чтобы получать из этого пользу. А для этого необходимо использовать актуальные алгоритмы и подходы, которые уже имеются.

Реализация современных алгоритмов поиска аномалий в данных может помочь в разных областях, таких как выявление отклонений в здоровье пациента, мошенничества в банковской сфере и др. Спрос на подобные сервисы в настоящее время высок и растёт с каждым днём.

\clearpage