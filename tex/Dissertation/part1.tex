\chapter{Анализ предметной области} \label{ch:ch1}

\section{Постановка задачи} \label{sec:ch1/sec1}

В настоящее время существует множество подходов к определению аномалий и выбросов в данных. Все они хорошо применимы в своей области. Именно поэтому следует в первую очередь проанализировать существующие алгоритмы и их сферы применимости. После чего предлагается объединить их в один сервис, чтобы упростить жизнь потенциальному пользователю, которому необходимо будет проанализировать большой объём данных.

\clearpage

\section{Объект, предмет и методы исследования} \label{sec:ch1/sec2}

\textbf{Объектом исследования} являются методы для экстраполяции временных рядов и поиска выбросов в данных, а также способы их реализации и области применения.

\textbf{Предметом исследования} является анализ существующих алгоритмов для обработки данных.

\textbf{Методы исследования} — анализ предметной области, анализ реализованных методов, написание программного кода и извлечение полезной информации из данных

\clearpage

\section{Актуальность выбранной темы} \label{sec:ch1/sec3}

Объём данных, которые появляются каждый день, растёт с экспоненциальной скоростью. Необходимо уметь работать с большими объёмами данных, чтобы получать из этого пользу. А для этого необходимо использовать актуальные алгоритмы и подходы, которые уже имеются.

Реализация современных алгоритмов поиска аномалий в данных может помочь в разных областях, таких как выявление отклонений в здоровье пациента, мошенничества в банковской сфере и др. Спрос на подобные сервисы в настоящее время высок и растёт с каждым днём.

\clearpage