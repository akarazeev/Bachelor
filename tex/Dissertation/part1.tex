\chapter{Анализ предметной области} \label{ch:ch1}

\section{Постановка задачи} \label{sec:ch1/sec1}

В настоящее время существует множество подходов к определению аномалий и выбросов в данных. Все они хорошо применимы в своей области. Именно поэтому следует в первую очередь проанализировать существующие алгоритмы и их сферы применимости. После чего предлагается объединить их в один сервис, чтобы упростить жизнь потенциальному пользователю, которому необходимо будет проанализировать большой объём данных.

\section{Результат работы алгоритмов} \label{sec:ch1/sec2}

Результатом работы алгоритма для поиска аномалий могут быть как \textbf{степени аномалии} (anomaly scores), так и \textbf{бинарные метки} (binary labels).

В случае, когда алгоритм выдаёт \textbf{степень аномалии}, под степенью понимается уровень вероятности того, что объект является выбросом (аномалией).

В случае \textbf{бинарных меток} алгоритм сразу указывает на нормальные (обычно обозначаемые как 0) и аномальные (обозначаемые как 1) данные. Несмотря на то, что некоторые алгоритмы детектирования аномалий возвращают бинарные метки напрямую, степени аномалий тоже могут быть переведены в бинарное представление. 0 или 1 содержат меньше информации, чем степень аномалии. Тем не менее, это конечный результат, по которому обычно принимается решение об аномальности объекта выборки.

\section{Классификация методов определения аномалий} \label{sec:ch1/sec3}

Большинство методов определения аномалий используют метки, по которым можно определить, является ли объект выборки нормальным или аномальным. Поиск или сбор размеченных данных, которые будут точными и хорошо описывать рассматриваемую проблемы, чтобы хорошо обучить алгоритмы, довольно сложно и дорого.

\noindent Обычно выделяют три типа методов поиска аномалий:

\begin{enumerate}
	\item \textbf{Supervised методы (обучение с учителем)}

Предполагается, что имеется доступ к обучающим данным с точными и репрезентативными метками для нормальных и аномальных объектов. В таком случае обычно разрабатывают предсказательную модель для обоих классов. После обучения на тренировочных данных к каждому объекту из тестовой выборки применяется алгоритм, чтобы определить класс объекта. \todo{Однако есть и проблема: получить точные и репрезентативные метки, особенно для аномалий, сложно. Поскольку аномалия определяется через смесь нескольких атрибутов. Такая ситуация довольно распространена в сценариях, таких как обнаружение мошенников в потоке информации в банковском секторе (сложно отличить мошенника от обычного пользователя только по действиям).}
	
	\item \textbf{Semi-supervised методы (обучение с частичным привлечение учителя)}

Предполагается, что имеются размеченные данные только для нормального класса. Так как для обучения таких алгоритмов не требуются размеченные аномальные данные, они имеют более широкое применение, чем supervised методы.
	\item \textbf{Unsupervised методы (обучение без учителя)}

Такие методы не требуют обучающих данных и поэтому наиболее широко используются. Unsupervised методы поиска аномалий могут нормальные данные из всех представленных и рассматривать отклонение от них как аномалию.
\end{enumerate}
Многие semi-supervised методы могут быть использованы для unsupervised случая. Например, с их помощью можно дополнительно семплировать объекты из выборки, если данных для обучения алгоритма попросту недостаточно.