\chapter{Сервис для анализа данных} \label{ch:ch3}

В этой главе будет рассмотрено устройство сервиса, которое было создано, чтобы упростить процесс анализа данных и поиска аномалий в них.

Ниже будут представлены основные сценарии работы с сервисом.

\begin{figure}[ht]
  \centering
  \includegraphics[width=\textwidth, height=\textheight, keepaspectratio] {demo_mainmenu}
  \caption{Главный экран сервиса. Есть возможность загрузить данные через окно загрузки (как продемонстрировано рисунке~\ref{fig:demo_uploadfile}), либо выбрать датасет из списка загруженных ранее.}
  \label{fig:demo_mainmenu}
\end{figure}

\begin{figure}[ht]
  \centering
  \includegraphics[width=\textwidth, height=\textheight, keepaspectratio] {demo_uploadfile}
  \caption{Пример экрана загрузки датасета на сервер.}
  \label{fig:demo_uploadfile}
\end{figure}

\begin{figure}[ht]
  \centering
  \includegraphics[width=\textwidth, height=\textheight, keepaspectratio] {demo_datasetloaded}
  \caption{После выбора файла на экране появляется таблица с некоторыми объектами из указанного датасета. С помощью меню, которое располагается справа, можно выбрать: (\textbf{i}) построение общего графика по датасету, на котором будут представлены данные, размерность которых была понижена с помощью метода t-SNE (продемонстрировано на рисунке~\ref{fig:demo_datasetoverview}), \textbf{(ii)} график зависимости одной из размерностей данных от другой (продемонстрировано на рисунке~\ref{fig:demo_plots}) и (\textbf{iii}) выбрать алгоритм из списка, с помощью которого будет осуществлён поиск выбросов в данных (продемонстрировано на рисунке~\ref{fig:demo_anomalydetection}).}
  \label{fig:demo_datasetloaded}
\end{figure}

\begin{figure}[ht]
  \centering
  \includegraphics[width=\textwidth, height=\textheight, keepaspectratio] {demo_datasetoverview}
  \caption{Представление данных из датасета после понижения размерности до dim=2 при помощи алгоритма t-SNE.}
  \label{fig:demo_datasetoverview}
\end{figure}

\begin{figure}[ht]
  \centering
  \includegraphics[width=\textwidth, height=\textheight, keepaspectratio] {demo_plots}
  \caption{Построение графика зависимости для указанных координат. На рисунке представлена зависимость максимального количества ударов в минуту от возраста пациентов (данные из датасета сердечных заболеваний).}
  \label{fig:demo_plots}
\end{figure}

\begin{figure}[ht]
  \centering
  \includegraphics[width=\textwidth, height=\textheight, keepaspectratio] {demo_anomalydetection}
  \caption{Поиск выбросов в данных. Датасет разбивается на две непересекающиеся части: на первой части выбранная модель обучается, а на второй -- проверяется качество обученной модели. Обычно, разделение происходит в соотношении 75\% и 25\% соответственно. Так снижается вероятность переобучения на данных, что позволяет избежать ухудшения обобщающей способности алгоритма.}
  \label{fig:demo_anomalydetection}
\end{figure}

На~рисунке~\ref{fig:demo_mainmenu} на~странице~\pageref{fig:demo_mainmenu} представлен главный экран сервиса.

%\begin{figure}[ht]
%  \begin{minipage}[ht]{0.49\linewidth}\centering
%    \includegraphics[width=0.5\linewidth]{demo_anomalydetection} \\ а)
%  \end{minipage}
%  \hfill
%  \begin{minipage}[ht]{0.49\linewidth}\centering
%    \includegraphics[width=0.5\linewidth]{demo_plots} \\ б)
%  \end{minipage}
%  \caption{Очень длинная подпись к изображению,
%      на котором представлены две фотографии Дональда Кнута}
%  \label{fig:demo_anomalydetection}
%\end{figure}

\clearpage