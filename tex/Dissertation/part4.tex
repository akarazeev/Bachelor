\chapter{Результаты} \label{ch:ch4}

\section{Выводы} \label{ch:ch4/sect1}

В ходе работы были проанализированы существующие библиотеки для языка программирования Python, которые упрощают работу с алгоритмами для поиска аномалий в данных.

Была обнаружена и исправлена ошибка в библиотеке PyOD\footnote{Pull Request \#108 в репозитории PyOD -- \url{https://github.com/yzhao062/pyod/pull/108}}.

Построен сервис для анализа данных и определения аномалий. Сервис развёрнут на удалённом сервере с операционной системой Ubuntu 18.04.2 LTS (Bionic Beaver)\footnote{\url{http://releases.ubuntu.com/18.04/}}.

В сервисе используется стандартный подход сервер-клиент. В качестве сервера выступает база данных PostgreSQL\footnote{\url{https://www.postgresql.org/docs/11/}}. С помощью библиотеки Flask\footnote{\url{http://flask.pocoo.org}} для языка программирования Python было построено веб-приложение -- на данный момент именно оно выступает в роли клиентского приложения, через который можно получить доступ к функционалу всего сервиса.

Помимо прочего, понадобилось дополнительное время, чтобы развернуть всю систему на сервере и добиться корректной работы. Сервис запущен по адресу \url{http://bit.ly/anomd19}.

\clearpage