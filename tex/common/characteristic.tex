
{\actuality} Обзор, введение в тему, обозначение места данной работы в
мировых исследованиях и~т.\:п., можно использовать ссылки на~другие
работы\ifnumequal{\value{bibliosel}}{1}{~\autocite{Gosele1999161}}{}
(если их~нет, то~в~автореферате
автоматически пропадёт раздел <<Список литературы>>). Внимание! Ссылки
на~другие работы в разделе общей характеристики работы можно
использовать только при использовании \verb!biblatex! (из-за технических
ограничений \verb!bibtex8!. Это связано с тем, что одна
и~та~же~характеристика используются и~в~тексте диссертации, и в
автореферате. В~последнем, согласно ГОСТ, должен присутствовать список
работ автора по~теме диссертации, а~\verb!bibtex8! не~умеет выводить в одном
файле два списка литературы).
При использовании \verb!biblatex! возможно использование исключительно
в~автореферате подстрочных ссылок
для других работ командой \verb!\autocite!, а~также цитирование
собственных работ командой \verb!\cite!. Для этого в~файле
\verb!Synopsis/setup.tex! необходимо присвоить положительное значение
счётчику \verb!\setcounter{usefootcite}{1}!.

Для генерации содержимого титульного листа автореферата, диссертации
и~презентации используются данные из файла \verb!common/data.tex!. Если,
например, вы меняете название диссертации, то оно автоматически
появится в~итоговых файлах после очередного запуска \LaTeX. Согласно
ГОСТ 7.0.11-2011 <<5.1.1 Титульный лист является первой страницей
диссертации, служит источником информации, необходимой для обработки и
поиска документа>>. Наличие логотипа организации на~титульном листе
упрощает обработку и~поиск, для этого разметите логотип вашей
организации в папке images в~формате PDF (лучше найти его в векторном
варианте, чтобы он хорошо смотрелся при печати) под именем
\verb!logo.pdf!. Настроить размер изображения с логотипом можно
в~соответствующих местах файлов \verb!title.tex!  отдельно для
диссертации и автореферата. Если вам логотип не~нужен, то просто
удалите файл с~логотипом.

\ifsynopsis
Этот абзац появляется только в~автореферате.
Для формирования блоков, которые будут обрабатываться только в~автореферате,
заведена проверка условия \verb!\!\verb!ifsynopsis!.
Значение условия задаётся в~основном файле документа (\verb!synopsis.tex! для
автореферата).
\else
Этот абзац появляется только в~диссертации.
Через проверку условия \verb!\!\verb!ifsynopsis!, задаваемого в~основном файле
документа (\verb!dissertation.tex! для диссертации), можно сделать новую
команду, обеспечивающую появление цитаты в~диссертации, но~не~в~автореферате.
\fi

% {\progress}
% Этот раздел должен быть отдельным структурным элементом по
% ГОСТ, но он, как правило, включается в описание актуальности
% темы. Нужен он отдельным структурынм элемементом или нет ---
% смотрите другие диссертации вашего совета, скорее всего не нужен.

{\aim} данной работы является \ldots

Для~достижения поставленной цели необходимо было решить следующие {\tasks}:
\begin{enumerate}
  \item Исследовать, разработать, вычислить и~т.\:д. и~т.\:п.
  \item Исследовать, разработать, вычислить и~т.\:д. и~т.\:п.
  \item Исследовать, разработать, вычислить и~т.\:д. и~т.\:п.
  \item Исследовать, разработать, вычислить и~т.\:д. и~т.\:п.
\end{enumerate}


{\novelty}
\begin{enumerate}
  \item Впервые \ldots
  \item Впервые \ldots
  \item Было выполнено оригинальное исследование \ldots
\end{enumerate}

{\influence} \ldots

{\methods} \ldots

{\defpositions}
\begin{enumerate}
  \item Первое положение
  \item Второе положение
  \item Третье положение
  \item Четвертое положение
\end{enumerate}
В папке Documents можно ознакомиться в решением совета из Томского ГУ
в~файле \verb+Def_positions.pdf+, где обоснованно даются рекомендации
по~формулировкам защищаемых положений.

{\reliability} полученных результатов обеспечивается \ldots \ Результаты находятся в соответствии с результатами, полученными другими авторами.


{\probation}
Основные результаты работы докладывались~на:
перечисление основных конференций, симпозиумов и~т.\:п.

{\contribution} Автор принимал активное участие \ldots

%\publications\ Основные результаты по теме диссертации изложены в ХХ печатных изданиях~\cite{Sokolov,Gaidaenko,Lermontov,Management},
%Х из которых изданы в журналах, рекомендованных ВАК~\cite{Sokolov,Gaidaenko},
%ХХ --- в тезисах докладов~\cite{Lermontov,Management}.

\ifnumequal{\value{bibliosel}}{0}{% Встроенная реализация с загрузкой файла через движок bibtex8
    \publications\ Основные результаты по теме диссертации изложены в XX печатных изданиях,
    X из которых изданы в журналах, рекомендованных ВАК,
    X "--- в тезисах докладов.%
}{% Реализация пакетом biblatex через движок biber
%Сделана отдельная секция, чтобы не отображались в списке цитированных материалов
    \begin{refsection}[vak,wos,scopus,papers,conf]% Подсчет и нумерация авторских работ. Засчитываются только те, которые были прописаны внутри \nocite{}.
        %Чтобы сменить порядок разделов в сгрупированном списке литературы необходимо перетасовать следующие три строчки, а также команды в разделе \newcommand*{\insertbiblioauthorgrouped} в файле biblio/biblatex.tex
        \printbibliography[heading=countauthorvak, env=countauthorvak, keyword=biblioauthorvak, section=1]%
        \printbibliography[heading=countauthorwos,env=countauthorwos, keyword=biblioauthorwos, section=1]%
        \printbibliography[heading=countauthorscopus,env=countauthorscopus, keyword=biblioauthorscopus, section=1]%
	\printbibliography[heading=countauthorconf, env=countauthorconf, keyword=biblioauthorconf, section=1]%
        \printbibliography[heading=countauthorothers, env=countauthorothers, keyword=biblioauthorothers, section=1]%
        \printbibliography[heading=countauthor, env=countauthor, keyword=biblioauthor, section=1]%
        \nocite{%Порядок перечисления в этом блоке определяет порядок вывода в списке публикаций автора
                vakbib1,vakbib2,%
		wosbib1,%
		scbib1,%
                confbib1,confbib2,%
                bib1,bib2,%
        }%
	\publications\ Основные результаты по теме диссертации изложены
	\setcounter{citeauthorscwostot}{\value{citeauthorscopus}} % вместе setcounter и addtocounter добавляют пробел между словами. По-этому они так раскиданы.
        в~\arabic{citeauthor}~печатных изданиях,
	\addtocounter{citeauthorscwostot}{\value{citeauthorwos}}
	\arabic{citeauthorvak} из которых изданы в журналах, рекомендованных ВАК\sloppy
	\ifnum \value{citeauthorscwostot}>0
	, \arabic{citeauthorscwostot} "--- в~периодических научных журналах, индексируемых Web of Science и Scopus\sloppy
	\fi
	\ifnum \value{citeauthorconf}>0
	, \arabic{citeauthorconf} "--- в~тезисах докладов.
	\else
	.
	\fi
    \end{refsection}
    \begin{refsection}[vak,wos,scopus,papers,conf]%Блок, позволяющий отобрать из всех работ автора наиболее значимые, и только их вывести в автореферате, но считать в блоке выше общее число работ
        \printbibliography[heading=countauthorvak, env=countauthorvak, keyword=biblioauthorvak, section=2]%
        \printbibliography[heading=countauthorwos, env=countauthorwos, keyword=biblioauthorwos, section=2]%
        \printbibliography[heading=countauthorscopus, env=countauthorscopus, keyword=biblioauthorscopus, section=2]%
        \printbibliography[heading=countauthorothers, env=countauthorothers, keyword=biblioauthorothers, section=2]%
        \printbibliography[heading=countauthorconf, env=countauthorconf, keyword=biblioauthorconf, section=2]%
        \printbibliography[heading=countauthor, env=countauthor, keyword=biblioauthor, section=2]%
        \nocite{vakbib2}%vak
        \nocite{bib1}%other
        \nocite{confbib1}%conf
    \end{refsection}
}
При использовании пакета \verb!biblatex! для автоматического подсчёта
количества публикаций автора по теме диссертации, необходимо
их~здесь перечислить с использованием команды \verb!\nocite!.
